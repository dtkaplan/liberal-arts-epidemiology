%% Author: J.R. Hebel and R.J. McCarter
%% Subject: Basic Health Statistics
%% Title: Study design

A cross-sectional study in 1976 revealed that the prevalence of oral
contraceptive use varied with age, as shown in the table.

\begin{center}
\begin{tabular}{cr}
Age & Prevalence\\\hline
15-19 & 15\%\\
20-24 & 25\%\\
25-29 & 22\%\\
30-34 & 15\%\\
35-39 & 7\%\\
40-44 & 3\%\\
\end{tabular}
\end{center}

The inference from these data that as women grow older they cease
using oral contraceptives is
\begin{MultipleChoice}
\wrong{Correct}
\wrong{Incorrect because a rate is necessary to support the
  observation}
\wrong{Incorrect because no control or comparison group is used}
\correct{Incorrect because a longitudinal design is needed}
\wrong{Incorrect because prevalence is used whereas incidence is necessary}
\end{MultipleChoice}


%% From J. R. Hebel and R.J. McCarter, "A Study Guide to Epidemiology and Biostatistics" 7/e page 125-126