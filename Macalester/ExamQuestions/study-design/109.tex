%% Author: Webb and Bain
%% Subject: Case control 
%% Title: Smoking and the Doll-Hill Case Control Study

Doll and Hill first evaluated the proposition that smoking was a risk
factor for lung cancer in a case-control study (Doll and Hill, 1950).
They found that, of 649 men with lung cancer (cases), 647 had smoked
at some time, compared with 622 of the 649 men without lung cancer.
[Evidently, smoking was very common in the British men from the 1940s
studied by Doll and Hill.]
\begin{enumerate}[(a)]
\item Draw up a clearly labelled and appropriate $2\times 2$ table to
  show these data.
\answerSpace{1.2in}
\item How many times more likely was a smoker to develop lung cancer
  than a non-smoker?
\answerSpace{.6in}
\item Calculate the proportion of lung cancers attributable to smoking
  among
\begin{enumerate}[(i)]
\item Smokers
\answerSpace{.6in}
\item The whole population.
\answerSpace{.6in}
\end{enumerate}
Say what the two measures in (i) and (ii) are called.
\answerSpace{.6in}

\item The 95\% confidence interval on the odds ratio is $3.2$ to
  $60.9$.  What does this mean about the strength of evidence that
  Doll and Hill had to associate lung cancer with smoking?
\answerSpace{.6in}

\end{enumerate}

%% from Webb and Bain, Essential Epidemiology 2/e, p. 1152, prob. 5.3