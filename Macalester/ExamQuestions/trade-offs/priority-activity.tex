
Discuss the following priority rating of health services (from the {\em Economics of Health Care} reading).

\begin{enumerate}
\item Treatments for children with life-threatening illnesses.
\item Special care and pain relief for people who are dying.
\item Preventive screening services and immunizations.
\item Surgery such as hip replacements to help people carry out everyday tasks.
\item District nursing and community services/care at home.
\item Psychiatric services for people with mental illnesses.
\item High technology surgery, organ transplants and procedures which treat life threatening conditions.
\item Health promotion/education services to help people lead healthy lives.
\item Intensive care for premature babies who weigh less than 680g with only a slight chance of survival.
\item Long stay hospital care for elderly people.
\item Treatment for infertility.
\item Treatment for people aged 75 and over with life threatening illness.
\end{enumerate}

{\bf Questions for discussion:} 
\begin{itemize}
\item Do you agree with the priority rating? How would you re-arrange it?  Does your group agree?  If not, what are the sources of disagreement?
\item What criteria do you use to decide on prioritization?  In answering this, you might want to focus on a few pairs:, e.g., 9 versus 11, or 5 versus 10, or 2 versus 12.
\end{itemize}
\bigskip

One possible way to think about prioritization is to imagine that you had
exactly 12 people, each of whom required treatment for exactly
one of the items listed above.  To establish the priority of the
items, put the 12 people in order from most pressing to least.

\bigskip

Of course the treatments might cost different amounts of money.  Would
this influence your prioritization?

\bigskip

{\bf Task:}   Once you have sorted out a prioritization, decide on a
policy process to determine how much money to spend on each one.  That
is, what would you look at to determine whether enough money is being
spent to honor the prioritization?




\newpage

\centerline{\bf Part II: Valuing a Year of Life}

\bigskip



\noindent Fill in the matrix below, assigning a score for ``quality of life'' to each of the situations.   
\medskip

``The term distress is an attempt to capture both the physical and mental effects of being ill.  This is very subjective but severe distress might mean considerable, continuous physical pain with perhaps a high level of anxiety and fear.''

\bigskip


\begin{center}
\begin{tabular}{|p{2in}|c|c|c|c|}\hline
 & \multicolumn{4}{c|}{\bf Distress Level}\\
{\bf Disability} & None & Mild & Moderate & Severe\\\hline
No disability & 1.000 & & &  \\\hline
Slight social disability &  & & & \\\hline
Severe social disability and/or slight impairment of performance at work.  Able to do all housework except very heavy tasks. & & & &  \\\hline
Choice of work or performance at work very severely limited.  Housewives and old people able to do light housework only but able to go out shopping. & & & & \\\hline
Unable to undertake any paid employment. Unable to continue any education.  Old people confined to home except for escorted outings and short walks and unable to do shopping.  Housewives able only to perform a few simple tasks. & & & & \\\hline
Confined to chair or to wheelchair or able to move around in the house only with support from an assistant. & & & & \\\hline
Confined to bed & & & & \\\hline
Unconscious & & & & \\\hline
\end{tabular}
\end{center}

\bigskip

{\bf Questions:}
\begin{itemize}
\item Do you agree with the ordering of the disability categories?  Should the ``Quality'' assigned decrease as you go down the rows of the table?

\item There are two dimensions in the table: Disability and Distress.  Are there other dimensions you would add?  

\item Implicit in the idea of a Quality Adjusted Life Year is that ``duration'' is an important issue: that helping a young person is substantially more valuable than helping an old person.  Do you agree?

\item What other functional disabilities (that is, inabilities to perform certain tasks) would you include in the rating of disabilities?  Examples: inability to drive, incontinence, ...

\end{itemize}


