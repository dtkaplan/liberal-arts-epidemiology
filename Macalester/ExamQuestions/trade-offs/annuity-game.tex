
You are 60 years old and need to put away money toward your retirement.  You will retire at age 65 and want to have an annual income of \$50,000.

\bigskip

How much money should you put away?

\bigskip

Here's a table of death probabilities for use in this problem:

\bigskip

\centerline{\begin{tabular}{l|cc}
 & \multicolumn{2}{c}{Probability}\\
Age & of dying & cumulatively \\\hline
65 & 0.1 & 0.1\\
70 & 0.1 & 0.2\\
75 & 0.1 & 0.3\\
80 & 0.3 & 0.6\\
85 & 0.1 & 0.7\\
90 & 0.1 & 0.8\\
95 & 0.1 & 0.9\\
100 & 0.05 & 0.95\\
105 & 0.05 & 1.00\\
\end{tabular}}

\bigskip

The payments start at 65, unless you happen to have died at 65.
Everybody is dead by 105, so 40 years the longest time you would
need to draw on your retirement.  The mean age at death is 82.25.  The
median is 80.

\begin{enumerate}
\item Write down how much money you will put aside at age 60 to
  support an annual income of \$50K per year for the rest of your life.
  
  
\item You will then be given, at random, an age of actual death.
  (Sorry!)  Calculate whether you had enough money in your savings to
  pay you the annual amount, and how many life-years you were without funds.
  
\item Score yourself.  Your score will be
  \begin{itemize}
  \item 2 million points minus the amount you put into savings.  (So
    if you put more into savings, your score is lower ...)
  \item Subtract from that 100,000 points times the number of life
    years you were without funds.
  \end{itemize}
  Example: You saved \$500,000 dollars and later found out that you
  died at age 80.  Your savings are enough for 10 years, and you
  retired at age 65, so you ran out of money at age 75.  This leaves
  you 5 years without funds.  Your score is $2,000,000 - 500,000 - 5
  \times 100,000 = 1,000,000$.

  
\end{enumerate}
\newpage

Note to Instructor:

Your actual age at death will be randomly selected from one of these
values, which correspond to the probabilities in the table.

{ \Large
\begin{Schunk}
\begin{Sinput}
> ages = c(rep(c(65, 70, 75, 80, 80, 80, 85, 90, 95), each = 4), 
+     rep(c(100, 105), each = 2))
> matrix(ages, nrow = 5)
\end{Sinput}
\begin{Soutput}
     [,1] [,2] [,3] [,4] [,5] [,6] [,7] [,8]
[1,]   65   70   75   80   80   85   90   95
[2,]   65   70   75   80   80   85   90  100
[3,]   65   70   80   80   80   85   95  100
[4,]   65   75   80   80   80   90   95  105
[5,]   70   75   80   80   85   90   95  105
\end{Soutput}
\end{Schunk}
}

