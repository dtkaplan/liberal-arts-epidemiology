%% Author: Penny Webb and Chris Bain
%% Subject: Basic Health Statistics
%% Title: prevalence and incidence

Two thousand women aged 55 years were given a health check and 100
were found to have high blood pressure.  Ten years later all 2000
women attended a second check and another 300 women had developed high
blood pressure.

\begin{enumerate}[(a)]

\item What was the prevalence of high blood pressure in the women (i)
  at age 55 and (ii) at age 65?
\answerSpace{.6in}
\begin{AnswerText}
Prevalence at age 55: 100/2000

Prevalence at age 65: 400/2000 (assuming the original 100 still had
high blood pressure.
\end{AnswerText}
\item How many women were ``at risk'' of developing high blood
  pressure at the start of the 10-year period?
\answerSpace{.6in}
\begin{AnswerText}
Since 100 people already had it, only the remaining 1900 were at risk for
developing high blood pressure.
\end{AnswerText}

\item What was the incidence of high blood pressure in these women?
  Is this a measure of cumulative incidence or an incidence rate?
\answerSpace{.6in}
\begin{AnswerText}
300 out of 1900 at risk got high blood pressure over the 10 year period.
That's a cumulative incidence of $\frac{300}{1900}/10$, or 15.8/1000
per year.

It's a cumulative incidence rather than an incidence rate because we
have not taken into account when each person got high blood pressure.
For an incidence rate, whenever someone gets high blood pressure, we
stop accumulating their years at risk.
\end{AnswerText}

\noindent Assume that, on average, each of the 300 women who developed
high blood pressure did so half-way through the ten year follow-up
period.
\item Calculate the total number of person-years at risk (of
  developing high blood pressure) during the 10 years.

\answerSpace{.6in}
\begin{AnswerText}
The assumption tells us that the 300 women who got high blood pressure
were (on average) at risk for only 5 years of the 10 year period.
That's 1500 person years-at risk.  In addition, the $1900-300=1600 $
women who were at risk but never got high blood pressure contribute
16000 person-years at risk.  Total: 17500 person-years at risk
\end{AnswerText}



\item What was the incidence rate of high blood pressure in these
  women?
\answerSpace{.6in}
\begin{AnswerText}
300 people per 17500 person-years at risk gives an incidence rate of
17.1/1000.  This is slightly higher than the cumulative incidence.
\end{AnswerText}


\end{enumerate}

%% From Webb and Bain, Essential Epidemiology, 2/e p. 68, prob. 2.2