%% Author: J.R. Hebel and R.J. McCarter
%% Subject: Basic Health Statistics
%% Title: Crude Mortality Rates

In 1970, the crude mortality rate (all causes) for Guyana (a
developing country)  was 6.8 per
1000 and for the United States it was 9.4 per 1000.

\begin{enumerate}[(a)]
\item Can the lower crude mortality rate in Guyana be explained by the
  fact that the United States has a larger population?  Explain your
  answer.
\answerSpace{.7in}
\begin{AnswerText}
No.  The mortality rate adjusts for the size of the population.
\end{AnswerText}
\item Give a probable explanation for the lower crude mortality
  rate in Guyana.
\answerSpace{.7in}
\begin{AnswerText}
Assuming that the lower rate in Guyana is not due to healthy living
conditions or good medical care (which would seem unlikely in a
developing country), a reasonable explanation is that the population
of Guyana tends to be younger than that of the United States.
\end{AnswerText}
\end{enumerate}


%% From J. R. Hebel and R.J. McCarter, "A Study Guide to Epidemiology and Biostatistics" 7/e page 21