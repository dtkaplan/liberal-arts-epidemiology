%% Author: J.R. Hebel and R.J. McCarter
%% Subject: Basic Health Statistics
%% Title: Crude Mortality Rates

Crude and age-adjusted mortality rates (per 100,000 people) from
``arteriosclerotic and degenerative heart diseases'' are shown for
Chile and the United States for 1967.  

\begin{center}
\begin{tabular}{lrr}
Country & Crude Rates & Age-Adjusted Rates\\\hline
Chile & 67.4 & 58.2\\
United States & 316.3 & 131.4\\
Ratio: US/Chile & 4.7 & 2.3\\
\end{tabular}
\end{center}


Which of the two rates is preferable for comparing the mortality rate
from heart disease in the two countries?  Why?  Why do the ratios of
the crude and age-adjusted rates for the two countries differ?

\begin{AnswerText}
In general, age-adjusted rates are much preferred to crude rates.  The
crude rate does not take into account the age distribution of the
population, which can vary markedly between countries (and between
eras within a country).  The crude and age-adjusted rates differ when
the age distribution differs.
\end{AnswerText}

%% From J. R. Hebel and R.J. McCarter, "A Study Guide to Epidemiology and Biostatistics" 7/e page 22