%% Author: J.R. Hebel and R.J. McCarter
%% Subject: Basic Health Statistics
%% Title: Crude Mortality Rates

A city contains 100,000 people (45,000 males and 55,000 females), and
1000 people die each year (600 males and 400 females).  There were 50
cases (40 males and 10 females) of lung cancer per year, of whom 45
died (36 males and 9 females). 

Compute the following:
\begin{enumerate}[(a)]

\item Crude mortality rate
\begin{AnswerText}
1000 deaths per year divided by the population of 100,000, that is,  1 per 1000.
\end{AnswerText}

\item Sex-specific mortality rate
\begin{AnswerText}
For females: 400 deaths per year divided by 55,000

For males: 600 deaths per year divided by 45,000
\end{AnswerText}

\item Cause-specific mortality rate for lung cancer
\begin{AnswerText}
Crude mortality rate for lung cancer: 45 deaths per year divided by
the population of  100,000
\end{AnswerText}

\item Case fatality rate for lung cancer
\begin{AnswerText}
45 deaths out of 50 cases: 45/50, that is, 90\% or 900 deaths per 1000 cases
\end{AnswerText}

\item Attributable Fraction for lung cancer
\begin{AnswerText}
Some of the people with lung cancer would have died anyways.  The
death rate for people without lung cancer is 955/99950 --- subtracting
out the people with lung cancer from the population along with their
deaths.  This is about 9.5 per 1000.  The death rate for lung cancer
is 900 per 1000.  So the death rate attributable to lung cancer is 890.5
per thousand.

\end{AnswerText}

\end{enumerate}

%% From J. R. Hebel and R.J. McCarter, "A Study Guide to Epidemiology and Biostatistics" 7/e page 22