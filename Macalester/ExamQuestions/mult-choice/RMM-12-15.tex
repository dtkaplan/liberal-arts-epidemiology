%% Author: RM Merrill
%% Topic: Screening
 
Screening has been associated with certain types of measurement bias.
Match the descriptions of bias resulting from screening with the names
of the types of bias.

\medskip
\centerline{\begin{tabular}{lcl}
A & & Lead-time bias\\
B & & Length bias\\
C & & Selection bias\\
D & & Overdiagnosis bias\\
\end{tabular}}
\medskip

\begin{enumerate}[(a)]
\item The screening test looks better than it actually is, because
  younger, healthier people are more likely to get the test. \SelectSetHoriz{C}{A,B,C,D,none}
\item Screening identifies an illness that would not have shown
  clinical signs before death from other causes. \SelectSetHoriz{D}{A,B,C,D,none}
\item Slow-progressing cases of disease with a better prognosis are
  more likely to be identified than faster-progressing cases of
  disease with poorer prognosis.  Thus, cases diagnosed through
  screening tend to have a better prognosis than the average of all
  cases. \SelectSetHoriz{B}{A,B,C,D,none}
\item Difference in the time between the date of diagnosis with 
  screening and the date of diagnosis without screening, 
  which if counted in the survival time of patients, will
  give a misleading picture of the benefits of treatment. \SelectSetHoriz{A}{A,B,C,D,none}
\end{enumerate}