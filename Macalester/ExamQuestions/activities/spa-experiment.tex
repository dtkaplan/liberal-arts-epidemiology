%% Author : Daniel Kaplan
%% Subject: Spread of Disease, index case
%% Title: Find the Source of the Infection

Here is the data from an experiment conducted in Mr. Dyar's 7th grade class at the Saint Paul Academy.

Each student was given a small cup of water.  But one student had a cup with a small amount of acid.  The water represented a healthy person, the acid a person with a transmissible disease.

The students had a sequence of 3 partners.   In interacting with each
partner they mixed the fluids in their cups and then split the mixed
fluid between them.  In this way, the acid was transmitted from one
student's cup to another.

All the interactions were synchronous, meaning that if student A was the first partner of student B, then B was the first partner of A as well, and similarly for second and third partners.  

{\bf After} interacting with all 3 partners, in order, each student's fluid
was tested to see if it contained the acid.  If so, it was deemed
``infected''.  The column labelled ``infected'' refers only to the
individual student in the left-most column.  To see if the partners
were also infected, you have to look up their rows in the table.


\begin{tabular}{l|lll|r}
Student & Partner 1 & Partner 2 & Partner 3 & Infected\\\hline
 Solomon & Cole & Brendan & Navodhya  & yes\\
 Joel &  Gus & Tommy  & Dalante  & no \\
 Blaire &  Taylor & Nina  &  Anna & yes \\
 Taylor &  Blaire & Bridget  & Nina  &  no\\ 
 Brendan &  Charlie & Solomon  & Cait  & yes  \\
 Charlie (H.) &  Brendan & Sam  & Bridget  & no\\ 
 Bridget &  Nina & Taylor  & Charlie  & no \\
 Tommy &  Navodhya & Joel  & Netta  & yes\\ 
 Nina &  Bridget & Blaire   & Taylor  & no   \\
 Anna &  Mr. Dyar & Netta  & Blaire  &  yes \\
 Gus &  Joel & Dalante  & Mr. Dyar  & no \\
 Cole &  Solomon & Mr. Dyar  & Sam  & no\\ 
 Sam (M.) &  Dalante & Charlie  & Cole  & no\\ 
 Dalante &  Sam & Gus  & Joel  & no\\
 Cait &  Netta & Navodhya  & Brendan  & yes \\
 Navodhya &  Tommy & Cait   &  Solomon & yes \\
Netta & Cait & Anna & Tommy & yes \\
Mr. Dyar & Anna & Cole & Gus & no\\\hline
\end{tabular}

\begin{enumerate}[(a)]
\item Who was the index case?  Explain your reasoning.

\TextEntry


\begin{AnswerText}
It's either Netta or Cait.  They are the only people who had all three of their partners infected.

There's no way to tell from the data whether it was Netta or Cait.  (In fact, it was Cait who was given the "infected" cup.)

\end{AnswerText}

\item Suppose that instead of having 3 partners, each student had 4 or 5 partners before being tested.  Would it be easier or harder to determine who was the index case?

\TextEntry

\end{enumerate}
