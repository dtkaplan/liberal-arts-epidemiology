%% Author: RM Merrill
%% Topic: Risk ratios

In a cohort study, a group of boys aged 8-15 years who were arrested because of substance abuse and identified as chemically dependent, were followed for 15 years.  These boys were also classified according to whether they had a history of sexual abuse.  Of interest was whether a history of sexual abuse was significantly associated with suicide attempt in these chemically dependent boys, the data for which are presented in the table:

\begin{center}
\begin{tabular}{lrr}
History of & \multicolumn{2}{c}{Suicide Attempt}\\
Sexual Abuse & Yes & No\\\hline
Yes & 11 & 34\\
No & 29 & 224\\
\end{tabular}
\end{center}

\begin{enumerate}[(a)]
\item What is the risk of suicide attempt for the chemically dependent boys with a history of sexual abuse?
\begin{MultipleChoice}
\correct{$11/45 = 244$ per thousand}
\wrong{$11/34 = 323$ per thousand}
\wrong{$29/253 = 115$ per thousand}
\wrong{$29/224 = 129$ per thousand}
\wrong{$30/258 = 116$ per thousand}
\end{MultipleChoice}
\item What is the risk of suicide attempt for the chemically dependent boys without a history of sexual abuse?
\begin{MultipleChoice}
\wrong{$11/45 = 244$ per thousand}
\wrong{$11/34 = 323$ per thousand}
\correct{$29/253 = 115$ per thousand}
\wrong{$29/224 = 129$ per thousand}
\wrong{$30/258 = 116$ per thousand}
\end{MultipleChoice}
\item What is the risk ratio?
\begin{MultipleChoice}
\wrong{$323/129$}
\correct{$244/115$}
\wrong{$115/129$}
\wrong{$11 \cdot 224 / 29 \cdot 34$}
\end{MultipleChoice}

\item Suppose eight boys with a history of sexual abuse were lost to follow-up and therefore not reported in the table.  In analyzing the situation, you estimate risk ratios for two extreme  hypothetical scenarios, one in which all the missing boys attempted suicide, and the other in which none of the missing boys attempted suicide.  You get two confidence intervals on the risk ratio: 
\begin{enumerate}[(i)]
\item $1.90$ to $5.14$ 
\item $0.97$ to $3.39$.
\end{enumerate}

Which hypothetical scenario does confidence interval (i) correspond to:
\begin{MultipleChoice}
\correct{All the missing boys committed suicide.}
\wrong{None of the missing boys committed suicide.}
\wrong{No way to know.}
\end{MultipleChoice}
\end{enumerate}