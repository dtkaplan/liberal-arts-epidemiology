%% Author: Julie Hoffer
%% Subject: Epi Vocabulary

Next to each definition, write the letter of the corresponding word from this set:

\bigskip

\centerline{\begin{tabular}{cl|cl|cl|cl} 
A  &  Antigen          & G &   Attack Rate & L &   Case-Control Study & Q &   Clinical Trial\\
B  &   Cohort Study & H &   Confounder & M &   Cross-Sectional Study & R &  Effect Modifier\\
C  &   Endemic         & I   &  Epidemic      &N   & Fomite   & S & Herd Immunity\\
D   & Immunoglobulin &J   & Incidence &O  &  Pandemic &T  &  Passive Immunity\\
E  &  Pathogenicity & K  &  Prevalence & P  &  Surveillance & V  &  Vector \\
F  &  Virulence & & & & & &\\
\end{tabular}}

\begin{enumerate}

\item The resistance of a group to an attack by a disease to which a large proportion of its members of the group are immune.



\item  A molecule capable of inducing an immune response and of being recognized by an antibody as a consequence of the immune response.


\item   A living animal, often an insect, that is capable of transmission of an infectious agent from an infected host to a susceptible host, resulting in infection.  %The agent may multiply in the vector or be carried without multiplication.


\item     The usual presence of disease within a geographical area or population group.


\item   The orderly collection, analysis and dissemination of information on incident disease cases.


\item    The number of new cases of a disease that occur during a specific period of time in a population at risk for developing the disease.  A time element such as person-years is not specified.


\item    The number of people at risk who develop an illness divided by the number of people at risk.


\item    A variable that influences the direction and magnitude of the association of interest in a study.


\item    A type of study design where a population is evaluated at a single point in time.  A causal relationship cannot be determined because the exposure and outcome are evaluated at the same time.


\item      The capability of an infectious agent to cause disease in a susceptible host.  
\end{enumerate}

