%% Author: Julie Hoffer
%% Subject: Case-control, odds ratio

A case-control study was conducted among men in the United States in order to find out whether a mother's use of hormones during pregnancy influenced her son's risk of developing testicular cancer later in life.  Investigators selected 500 cases who were hospitalized for testicular cancer and 1000 controls.  The study found that 90 cases' mothers and 50 controls' mothers had used hormones during pregnancy.

\begin{enumerate}	
\item Set up the 2 x 2 table for these data.

\answerSpace{3cm}

\begin{AnswerText}
\begin{tabular}{l|cc}
Exposure &	Cases& Controls\\\hline
Hormones (E+) &	90 &	50\\
No Hormones (E-) &	410 &	950\\
\end{tabular}
\end{AnswerText}

       
\item Calculate the odds ratio.
\answerSpace{2cm}

\begin{AnswerText}
 OR  $\frac{90 \times 950}{50\times 410}  =  4.2$
\end{AnswerText}

\item What is the purpose of the control group in a case control study.
\answerSpace{2cm}

\begin{AnswerText}
 The control group provides information on the exposure distribution in the population that produced the cases 
\end{AnswerText}

\item Describe an advantage and disadvantage of a case control study.
\answerSpace{3cm}
      
\begin{AnswerText}
\noindent  Advantages:
\begin{itemize}  
\item              Less time and money that cohort or clinical trial 
\item              Good design for rare diseases or diseases with long incubation and latent periods
\item              Can provide information on a large number of risk factors (exposures)
\end{itemize}


\noindent Disadvantages:
\begin{itemize}
\item        Possibility of bias is increased over cohort or clinical trials
\item         Data is retrospective, temporal relationship may not be determined
\end{itemize}

\end{AnswerText}

\end{enumerate}
