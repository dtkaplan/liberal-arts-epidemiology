%% Author: Julie Hoffer
%% Subject: Study design

\begin{enumerate}
\item State whether or not a {\bf cohort study} is well suited for each of the following scenarios.
\begin{enumerate}[(a)]
\item When little is known about a rare disease  \SelectSetHoriz{No}{Yes,No} 
\item When little is known about a rare exposure   \SelectSetHoriz{Yes}{Yes,No} 
\item When the study population will be difficult to follow  \SelectSetHoriz{No}{Yes,No} 
\item When you want to learn about multiple effects of an exposure   \SelectSetHoriz{Yes}{Yes,No} 
\end{enumerate}

\item In a case control study, which of the following is true?
\begin{MultipleChoice}
\wrong{The proportion of cases with the exposure is compared with the proportion of controls with the exposure.}
\wrong{Disease rates are compared for people with the factor of interest and for people without the factor of interest.}
\wrong{The investigator may choose to have multiple comparison groups}
\wrong{Recall bias is a potential problem}
\correct{A, C, and D}
\end{MultipleChoice}

\item In a cohort study, the advantage of starting by selecting a defined population for study before any of its members become exposed, rather than starting by selecting exposed and nonexposed individuals, is that:
\begin{MultipleChoice}
\wrong{The study can be completed more rapidly}
\wrong{A number of outcomes can be studied simultaneously}
\correct{A number of exposures can be studied simultaneously}
\wrong{The study will cost less to carry out}
\wrong{Both A and D}
\end{MultipleChoice}

\item Which of the following in not an advantage of a (prospective) cohort study?
\begin{MultipleChoice}
\correct{It usually costs less that a case-control study}
\wrong{Precise measurement of exposure is possible}
\wrong{Incidence rates can be calculated}
\wrong{Recall bias is minimized compared with a case-control study}
\wrong{Many disease outcomes can be studied simultaneously}
\end{MultipleChoice}
\end{enumerate}
