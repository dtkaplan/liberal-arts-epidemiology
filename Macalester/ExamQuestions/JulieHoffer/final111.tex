%% Author: Julie Hoffer
%% Subject: Reading a Study Abstract

Here is the abstract from a {\em New England Journal of Medicine} article 

``The Long-Term Effects of exposure to Low Doses of Lead in Childhood, an 11-year Follow-up Report'' {\em NEJM} 332:83 (1990)

\noindent To determine whether the effects of low-level lead exposure persist, we reexamined 132 of 270 young adults who had initially been studied as primary school children in 1975 through 1978.  In the earlier study, neurobehavior functioning was found to be inversely related to dentin lead levels\footnote{Lead levels in the body of a tooth}.  As compared with those we restudied, the other 138 subjects had had lower IQ scores and poorer teacher' ratings of classroom behavior.  When the 132 subjects were reexamined in 1988, impairment in neurobehavioral function was still found to be related to the lead content of teeth shed at the ages of six and seven.  The young people with dentin lead levels $>$ 20 ppm had a markedly higher risk of dropping out of high school; adjusted risk ratio, 7.4,  with a 95\% CI of [1.4 to 40.7] and of having reading disability; risk ratio, 5.8, with a 95\% CI [1.7 to 19.7] as compared with those with dentin lead levels $<$ 10 ppm.  Higher lead levels in childhood were also significantly associated with lower class standing in high school, increased absenteeism, lower vocabulary and grammatical reasoning scores, poorer hand-eye coordination, longer reaction times, and slower finger tapping.  No significant associations were found with the results of 10 other tests of neurobehavioral functioning.  Lead levels were inversely related to self-reports or minor delinquent activity.  We conclude that exposure to lead in childhood is associated with deficits in central nervous system functioning that persist into young adulthood.

\begin{enumerate}

\item What was the objective of the study?

\answerSpace{3cm}

\begin{AnswerText}
The purpose of the study was to evaluate neuropsychological and academic performance in young adulthood from a sample of primary grade school children who were first studied in 1975 to 1978.
\end{AnswerText}

\item What was the primary exposure of interest?
\answerSpace{1cm}

\begin{AnswerText}
The primary exposure was dentin lead level from young children.
\end{AnswerText}

\item What was the primary outcome(s) of interest?

\answerSpace{1cm}

\begin{AnswerText}
Several neuropsychological and academic performance outcomes were used.  
Hand-eye coordination, reaction time, vocabulary, grammatical reasoning, reading, dropping out of high school, class standing, absenteeism.
\end{AnswerText}

\item What type of study was conducted?

\answerSpace{1cm}

\begin{AnswerText}
Cohort or Retrospective Cohort
\end{AnswerText}

% \item Describe the source of the study population and sample size.
%
% \answerSpace{2cm}
%
% \begin{AnswerText}
% Population:  Children who were initially studied in 1975-1978 

% Sample Size:  Initial study 270 children, Current study = 132
% \end{AnswerText}

\item What types of bias could have occurred in this study?

\answerSpace{1cm}

\begin{AnswerText}
Selection bias:  children (currently in school – able to participate in regular classroom), children able to provide lost teeth for study.  Current study selection bias, children could not have moved.  Parents would have to provide informed consent.
\end{AnswerText}

\item What measures of association were presented?

\answerSpace{2cm}

\begin{AnswerText}
Relative Risk
\end{AnswerText}

\item What measures of statistical stability/reliability/confidence were reported in this study?

\answerSpace{1cm}

\begin{AnswerText}
Confidence Interval
\end{AnswerText}

% \item What were the major results of this study?
%
% \begin{AnswerText}
%Impairment in neurobehavioral functioning in young adults was related to dentin lead levels from earlier in childhood.  7.4-fold increased risk of dropping out of high school, and a 5.8-fold increased risk of having a reading disability among children with $> 20$ ppm dentin lead level compared to $< 10$ ppm dentin lead levels.
% \end{AnswerText}

% \item What was the authors' main conclusion?
%
% \begin{AnswerText}
% Exposure to lead in childhood is associated with deficits in central nervous system functioning that persist into young adulthood.
% \end{AnswerText}

% \item What larger population can the results of this study be generalized?
%
% \begin{AnswerText}
% The results can be generalized to other children in developed countries of similar socioeconomic status with similar lead levels.
% \end{AnswerText}

 

\end{enumerate}

