%% Author: Julie Hoffer
%% Subject: Screening

100,000 men were screened for prostate cancer for the first time.  Of these, 4000 men had a positive result on the screening blood test; of those who tested positive, 800 had a biopsy indicating a diagnosis of prostate cancer.  Among the remaining 96,000 men who screened negative, 100 developed prostate cancer within the following year and were assumed to be false negatives to the screen.

\begin{enumerate}[(a)]
\item Create a 2 x 2 table for the diagnostic test and prostate cancer.
\answerSpace{3cm}

\begin{AnswerText}
\centerline{\begin{tabular}{lrrr}
 & Prostate & Prostate & \\
 & 	 Cancer +	&  Cancer -	& Total\\\hline
Blood Test + &	800 &	3,200 &	4,000\\
Blood Test - &	100 &	95,900 &	96,000\\\hline
Total &	900 &	99,100 & 	100,000\\ 
\end{tabular}}
\end{AnswerText}

\item What is the prevalence of prostate cancer in this population?

\begin{AnswerText}
$900/100,000 = 0.9$\%
\end{AnswerText}

\item Calculate and interpret the sensitivity of this screening test.

\begin{AnswerText}
Sensitivity = $800/900$ or 89\%

 89\% of the men with prostate cancer screened positive with this test.
\end{AnswerText}

\item  Calculate and interpret the specificity of this screening test.

\begin{AnswerText}
 Specificity $ = 95,900/99,100$ or $96.8$\%
96.8\% of the men without prostate cancer screened negative.
\end{AnswerText}

\item There is a widespread assumption that screening a population to detect the early stages of disease is always beneficial.  However, there are risks and costs that must be weighed against the benefits of screening.  Briefly describe one hidden cost of screening for prostate cancer in this example. 

\begin{AnswerText}
11\% of men with prostate cancer will not be detected by this screening test.
If a biopsy was performed on all men who screened positive, 3,200 men would have had an unnecessary biopsy.
\end{AnswerText}

\end{enumerate}

