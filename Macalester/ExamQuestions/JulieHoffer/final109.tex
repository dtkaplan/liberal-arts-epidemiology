%% Author: Julie Hoffer
%% Subject: Exam Multiple Choice

\begin{enumerate}
\item The identification of an unrecognized disease by the application of tests, examinations or other procedures before the onset of symptoms.

\begin{MultipleChoice}
\wrong{Prevention}
\correct{Screening}
\wrong{Sensitivity}
\wrong{Lead time}
\end{MultipleChoice}

\item Which of the following is {\bf not} on of Bradford Hill's Guidelines for assessing causation.

\begin{MultipleChoice}
\wrong{Strength of Association}
\wrong{Consistency}
\correct{Sufficient Cause}
\wrong{Temporality}
\wrong{Biological Gradient}
\end{MultipleChoice}

\item The following is an example of secondary prevention:

\begin{MultipleChoice}
\wrong{Careful control of insulin levels and patient education to prevent circulatory complications among patients with diabetes}
\wrong{Clear air, water and food supply}
\wrong{Immunization for seasonal influenza}
\correct{Screening for HIV combined with early use of antiretroviral therapy}
\end{MultipleChoice}


\item What are the two main types of morbidity rates commonly used in epidemiology?

\begin{MultipleChoice}
\wrong{point prevalence rate and period prevalence rate}
\correct{incidence rate and prevalence rate}
\wrong{crude death rate and cause-specific death rate}
\wrong{relative risk and odds ratio}
\end{MultipleChoice}

\item Which statistic is most often used to compare risks in a case-control study?

\begin{MultipleChoice}
\correct{Odds ratio}
\wrong{Prevalence}
\wrong{Relative Risk}
\wrong{All of the above}
\end{MultipleChoice}

\item A rate ratio:

\begin{MultipleChoice}
\wrong{Compares the prevalence of a disease over time}
\correct{Compares the rates of disease in two groups that have different characteristics}
\wrong{Estimates the lifetime chance of developing a disease}
\wrong{Predicts the number of deaths from a given number of disease cases}
\end{MultipleChoice}


\end{enumerate}