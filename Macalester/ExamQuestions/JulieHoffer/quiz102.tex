%% Author: Julie Hoffer
%% Subject: Odds Ratios and Confidence Intervals

In the following report, the exposure of interest is pre-packaged salad, the cases are {\em E. coli} O157:H7 infection.  The crude odds ratio and its 95\% confidence interval are given along with age-stratified odds ratios and confidence intervals.  

\bigskip

\centerline{\begin{tabular}{lrc}
\multicolumn{3}{c}{Pre-packaged Salad}\\\hline
 & Odds Ratio & CI at 95\%\\\hline
Crude & 1.41 & 1.10-1.82\\\hline
Age Group & & \\\hline
0	   & 3.60 & 	0.21-62.8\\
1-9     & 1.39 &	0.81-2.41\\
10-19 & 0.94 &	0.51-1.73\\
20-29 & 1.62 &	0.84-3.16\\
30-39 & 4.70 &	1.98-11.2\\
40-49 & 1.18 &	0.52-2.70\\
50-59 & 1.65 &	0.73-3.74\\
60+    & 1.53 &	0.77-3.04\\
\end{tabular}}

\begin{enumerate}[(a)]
\item Does the confidence interval for the crude association between prepackage salad and incident Escherichia coli O157:H7 indicate there is a likely association between the exposure and disease?

\answerSpace{2cm}

\item For the 60+ age group, the odds ratio is 1.53 but the center of the confidence interval is approximately 1.9.  Why the difference?

\answerSpace{2cm}

\item Give a plausible explanation for why the confidence interval for the Age Group 0 is so much wider than for the other groups.

\answerSpace{2cm}

$\ $
\end{enumerate}

