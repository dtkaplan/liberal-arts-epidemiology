%% Author: Julie Hoffer
%% Subject: Study Design

Based on the following information, determine the type of study.

\begin{enumerate}
\item In a study begun in 1965, a group of 3,000 adults in Baltimore were asked about alcohol consumption.  The occurrence of cases of cancer between 1981 and 1995 was studied in this group.  

\begin{MultipleChoice}
\wrong{Case-Control}
\wrong{Clinical Trial}
\correct{Cohort}
\wrong{Cross-sectional}
\wrong{Retrospective Cohort}
\end{MultipleChoice}

\item In a small pilot study, 12 women with endometrial cancer (cancer of the uterus) and 12 women with no apparent disease were contacted and asked whether they had ever used estrogen.  Each woman with cancer was matched by age, race, weight and parity to a woman without disease.

\begin{MultipleChoice}
\correct{Case-Control}
\wrong{Clinical Trial}
\wrong{Cohort}
\wrong{Cross-sectional}
\wrong{Retrospective Cohort}
\end{MultipleChoice}

\item The physical examination records of the entire incoming freshmen class of 1935 at the University of Minnesota were examined in 1977 to see if their recorded height and weight at the time of admission to the university was related to the development of coronary heart disease by 1986.

\begin{MultipleChoice}
\wrong{Case-Control}
\wrong{Clinical Trial}
\wrong{Cohort}
\wrong{Cross-sectional}
\correct{Retrospective Cohort}
\end{MultipleChoice}

\item In the 1940s, Sir Norman Gregg, an Australian ophthalmologist, observed a number of infants and young children in his ophthalmology practice who presented with an unusual form of cataract.  Gregg noted that these children had been in utero during the time of a rubella outbreak.  He suggested that there was an association between prenatal rubella exposure and the development of the unusual cataracts.  Keep in mind that at that time there was no knowledge that a virus could be teratogenic.  

\begin{MultipleChoice}
\correct{Case-Control}
\wrong{Clinical Trial}
\wrong{Cohort}
\wrong{Cross-sectional}
\wrong{Retrospective Cohort}
\end{MultipleChoice}

\item About 6\% of adults older that 30 years of age and 12\% of adults older than 65 years of age have significant knee pain as a result of osteoarthritis.  180 veterans were randomized to a group receiving arthroscopic debridement (59), a group receiving arthroscopic lavage (61) or a placebo group receiving a sham intervention (60). Outcomes of level of knee pain and functional level were measured for a 2-year period of time.

\begin{MultipleChoice}
\wrong{Case-Control}
\correct{Clinical Trial}
\wrong{Cohort}
\wrong{Cross-sectional}
\wrong{Retrospective Cohort}
\end{MultipleChoice}

\item Residents of three villages with three different types of water supply were asked to participate in a survey to identify cholera carriers.  Because several cholera deaths had occurred recently, virtually everyone present at the time underwent examination.  The proportion of residents is each village who were carriers was computed and compared.

\begin{MultipleChoice}
\wrong{Case-Control}
\wrong{Clinical Trial}
\wrong{Cohort}
\correct{Cross-sectional}
\wrong{Retrospective Cohort}
\end{MultipleChoice}
 
\end{enumerate}