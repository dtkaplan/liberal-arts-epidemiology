%% Author: Julie Hoffer
%% Subject: Odds Ratios

On September 18, 2007 an illness complaint associated with a church festival was received on the Minnesota foodborne illness hotline.   The initial investigation was conducted in response to the complaint. 
The investigation revealed more than 20 reports of gastrointestinal illness from multiple households. 

  A food questionnaire and illness history were completed by a sample of the roughly 1000 individuals who attended and ate at the event.  The results of the questionnaire are as follows:

\bigskip

\centerline{\begin{tabular}{l|rr|rr}
 & \multicolumn{2}{c|}{People who} & \multicolumn{2}{c}{People who}\\
 & \multicolumn{2}{c|}{ate Item} & \multicolumn{2}{c}{did not Eat Item}\\\hline
Food        &	Sick & Not Sick & Sick & Not Sick\\\hline
Pork Roast&	21 & 1 &	8 &	13\\
Bratwurst	&	1 & 20 &	5 &	16\\
Gravy	&	22 & 1 &	7 &	14\\
Dressing	&	12 & 11&	7 &	14\\
Coleslaw	&	16 & 7 &	4 &	17\\
\end{tabular}}

\begin{enumerate}[(a)]
\item What type of study would you design to evaluate this problem?  What aspects of the situation shape your decision?

\answerSpace{2cm}

\item For each food item, write down the odds of getting sick for people who
  ate that item.  (You can write it simply as a ratio --- you don't
  need to convert to a decimal.)

\answerSpace{2cm}

\item	Calculate the odds ratio for coleslaw and for bratwurst.

\answerSpace{2cm}

\item	Identify the food(s) that are strongly associated with the
  outbreak. 

\answerSpace{2cm}
\end{enumerate}

