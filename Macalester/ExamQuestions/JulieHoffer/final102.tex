%% Author: Julie Hoffer
%% Subject: Risk ratio, relative risk

The following study looked at 20,000 individuals.  The exposure of interest was hypertension, and the outcome of interest was myocardial infarction.  The sample size for each level of exposure is 10,000.  Of the individuals who had a myocardial infarction, 180 had hypertension and 30 were normal for hypertension status.  

\begin{enumerate}
\item Set up the table for these data:

\answerSpace{4cm}
\begin{AnswerText}
\begin{tabular}{l|cc|l}
Exposure &	MI+ & 	MI- & 	Total \\\hline
Hypertension (E+) & 	180	& 9820 &	10,000\\
No Hypertension (E-) &	30	& 9970 &	10,000\\\hline
Total & 	210	& 19,790 & 	20,000\\
\end{tabular}
\end{AnswerText}


\item Calculate the relative risk of having a myocardial infarction in hypertensive individuals compared to non-hypertensive individuals.

\answerSpace{2cm}

\begin{AnswerText}
RR $= \frac{180/10,000}{30/10,000} = 6.00 $
\end{AnswerText}

\item How much of the risk of myocardial infarction is due to being hypertensive?
\answerSpace{3cm}


\begin{AnswerText}
Risk Hypertensive $= 180/10,000 = 1.8$\%

Risk No Hypertension $= 30/10,000 = 0.3$\%
 
Risk attributable to hypertension $ 1.8 – 0.3 = 1.5$\%
\end{AnswerText}

\item What is the study design? (e.g., case-control, cross-over, cohort, cross-sectional, ...)
\answerSpace{1cm}

\begin{AnswerText}
Cohort
\end{AnswerText}



\end{enumerate}